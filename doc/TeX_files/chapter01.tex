\chapter{Introducción}
\Huge{Editor de Geometry Dash retrofunk}

\Huge{Como leer esta documentación} \\
\LARGE{primero, hay cosas que tener en cuenta:}
\begin{enumerate}  
\item Esto es documentación, no tutorial,
\item No documentación para vivir en EEUU.
\item Como te has podido dar cuenta, soy un gracioso.
\item Esta documentacion va a estar desactualizada con cada actualización.
\end{enumerate}

\large{Tipos de letra} \\
\normalsize {
Si esta escrito en alguna fuente, significan cosas:

{\ttfamily Debes modificar lo que va con esta fuente.} \\

{\ttfamily  	{\em Lo que tiene esta fuente es una nota.  } }

Si no logras diferenciarla, en el PDF esta muy distinta, pero no recomiendo mucho el PDF.
\\\\
Tambien PUEDE haber formulas o operaciones matemáticas, se ven algo asi:}
\Huge{
$ \frac{13}{37} + (\frac{32}{32} \cdot \frac{-13}{\sqrt[13]{37}} / \frac{1*3*2}{4/3*2}) $ 
}
\\
\normalsize{Envia el resultado de esa operación a qorg@vxempire.xyz para recibir absolutamente nada!}
\\
\\
\\
\LARGE{Paso de que me hagan preguntas, asi que respondo aquí mismamente}
\begin{enumerate}
\item¿Por que usas LaTeX? Es un programa del año de la patata. \\
  Por que soy un purista, además, ocuparia muchisimo menos espacio en el repositorio que si estuviera en pdf o en html
  Ademas, en el PDF que no se por que no esta yendo muy bien, tiene una calidad de salida muy buena.
\item¿Por que has hecho esto?
  \\
\href{https://youtu.be/JkhX5W7JoWI?t=1m7s}{Respuesta} \\
\item¿Se puede redistribuir esta documentación libremente? \\
  Sí, mientras se cumpla la licencia \href{https://creativecommons.org/licenses/by/4.0/}{CC BY 4.0}
\\ Las imagenes de las operaciones matemáticas tambien estan bajo dicha licencia.
\end{enumerate}
